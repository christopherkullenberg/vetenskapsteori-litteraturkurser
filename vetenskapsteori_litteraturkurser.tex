\documentclass[12pt,a4paper]{article}
\usepackage[utf8]{inputenc}

\usepackage[english]{babel}
\usepackage{amsmath}
\usepackage{amsfonts}
\usepackage{amssymb}
\usepackage{graphicx}
\usepackage[style=apa,backend=biber,natbib=true]{biblatex}
\usepackage{bibentry}

\addbibresource{/home/anon/Dropbox/klassiskmodern/bibliography.bib}


\begin{document}


\begin{Huge}
Vetenskapsteori: Litteraturkurser
\end{Huge}

\medskip 
{\Large Områdesbeskrivningar och litteraturförslag}


\makeatletter
\@starttoc{toc}
\makeatother


\section{Klassisk vetenskapsteori}

\subsection{Empirism (Empiricism)}

 Detta delområde behandlar empiristiskt inriktade vetenskapsteorier och dess centrala frågeställningar. Till exempel: Vad är en empirisk observation? Vilken roll spelar de mänskliga sinnena? Vilken roll spelar instrument och experiment? I vilken utsträckning kan vi nå säker kunskap genom empiriska observationer eller experiment? Hur når vi kunskap om orsak och verkan? Var går gränserna för vad som kan bevisas empiriskt? Studenten kan även fördjupa sig i empirismens historiska utveckling från upplysningen fram till idag samt empirismens påverkan på den vetenskapsteoretiska uppfattningen om hur vetenskaperna utvecklas historiskt. Studenten kan fördjupa sig inom ett brett område för empirismen, vilket inkluderar positivism och logisk empirism. 
 
\noindent \textbf{Litteraturförslag}

 \fullcite{baconNewOrganon2000}
 
 \fullcite{carnapLogicalStructureWorld1967}
 
 \fullcite{carusCarnapTwentiethcenturyThought2007}
 
 \fullcite{humeEnquiryConcerningHuman2007}
 
 \fullcite{reichenbachExperiencePredictionAnalysis1938}
 
 \fullcite{sigmundWienkretsenFilosofiVid2017}
 
 \fullcite{wittgensteinTractatusLogicophilosophicus2013}
 

 
\subsection{Rationalism (Rationalism)} 

Detta delområde behandlar rationalistiskt inriktade vetenskapsteorier, det vill säga vilken roll som det mänskliga förnuftet och tänkandet spelar inom vetenskaperna. Exempel på frågeställningar: Hur kan det mänskliga förnuftet skapa en grund för vetenskaperna? Är det mänskliga förnuftet en inneboende förmåga eller utvecklas det över tid? Vilken roll spelar matematiska, geometriska och logiska utvecklingar för de empiriska vetenskaperna? Studenten kan fördjupa sig inom ett brett definierat område av rationalistiska ansatser från upplysningen fram till idag, vilket även inkluderar ansatser som kombinerar rationalism med empirism och pragmatism. 

\noindent \textbf{Litteraturförslag}

 \fullcite{cassirerSubstanceFunctionEinstein1923}

 \fullcite{descartesDiscourseMethodMeditations1998}

 \fullcite{fregeFoundationsArithmeticLogicomathematical1980}

 \fullcite{kantKritikAvDet2004}
 

\subsection{Pragmatism (Pragmatism)}
Detta delområde behandlar den pragmatistiskt inriktade vetenskapsteorin, från den Amerikanska pragmatismen fram till nutida utvecklingar och ansatser. Exempel på frågeställningar: Vilken roll spelar vetenskapernas praktiska användningsområden i förhållande till dess grundforskning? Vilka konsekvenser har den vetenskapliga praktiken för vetenskaplig bevisföring? Hur utvecklas vetenskaplig kunskap historiskt? Vilken roll spelar vetenskapssamhället i relation till forskningsprocessen?

\noindent \textbf{Litteraturförslag}

 \fullcite{}
 
 \fullcite{}
  
 \fullcite{}

\subsection{Hermeneutik (Hermeneutics)}
Detta delområde behandlar hermeneutikens vetenskapsteoretiska utgångspunkter. Exempel på frågeställningar: Hur skapas mening? Vilken roll spelar det tolkande subjektets förförståelse? Hur kan man skapa intersubjektivt giltig kunskap? Hur förstår vi historien utifrån samtiden? Studenten ges möjlighet att fördjupa sig både genom historiskt kanoniserade verk inom den sekulära hermeneutiken, samt inom samtida hermeneutiska frågeställningar som inkluderar en problematisering av den normativa uppfattningen av det tolkande subjektet. 

\noindent \textbf{Litteraturförslag}

 \fullcite{diltheyIntroductionHumanSciences1989}

 \fullcite{gadamerSanningOchMetod1997}
 
 \fullcite{heideggerVaraOchTid2019}
 
 \fullcite{rickertLimitsConceptFormation1986}
  
 \fullcite{schleiermacherSchleiermacherHermeneuticsCriticism1998}

\subsection{Fenomenologi (Phenomenology)} Detta område behandlar fenomenologiska perspektiv på vetenskaplig kunskapsproduktion. Exempel på frågeställningar är: vad är förhållandet mellan varseblivning och verklighet? Vad är relationen mellan subjekt och objekt? Vilken roll spelar den kroppsliga erfarenheten? Hur skapas mening genom erfarenhet? Studenten kan fördjupa sig inom ramen för nittonhundratalets utveckling inom fenomenologin, vilket även innefattar kritiska ansatser. 


\noindent \textbf{Litteraturförslag}

 \fullcite{}
 
 \fullcite{}
  
 \fullcite{}

\subsection{Falsifikationism (Falsificationism)}
Detta område behandlar falsifikationistiska perspektiv på vetenskaplig kunskapsproduktion. Exempel på frågeställningar: Hur kan vi upprätta en demarkation mellan vetenskap och icke-vetenskap? Hur kan vi beskriva vetenskapernas ökade kunskap utifrån en falsifikationistisk grund? Vilken roll spelar motbevis i den vetenskapliga praktiken? Vilka är gränserna för en falsifikationistisk ansats? Var går gränsen mellan context of discovery och context of justification? Studenten kan fördjupa sig inom nittonhundratalets hela utveckling inom falsifikationismen, vilket även inkluderar historiska perspektiv. 


\noindent \textbf{Litteraturförslag}

 \fullcite{popperLogicScientificDiscovery2002}
 
 \fullcite{}
  
 \fullcite{}

\subsection{Vetenskapssociologi (Sociology of science)}
Detta område behandlar sociologiska perspektiv på vetenskaplig kunskapsproduktion. Exempel på frågeställningar: Vilken roll spelar normer och värderingar inom forskningen? Hur påverkas forskningen av samhällsförändringar? Vilka sociala faktorer påverkar forskningens genomförande och på vilket sätt? Studenten kan fördjupa sig inom den utveckling som sker från den klassiska sociologin fram till startpunkten för de moderna vetenskapssociologiska ansatserna. 


\noindent \textbf{Litteraturförslag}

 \fullcite{}
 
 \fullcite{}
  
 \fullcite{}

\subsection{Feministisk vetenskapsteori (Feminist theory of science)} 
Detta område behandlar feministiska perspektiv på vetenskap. Exempel på frågeställningar: Vilken roll spelar forskarens kön för hur forskningsfrågor utformas? Vilken roll spelar kön i hur vi beskriver forskningens historiska framsteg? Vilken inverkan har kön på forskares karriärmöjligheter? Hur påverkas det vetenskapliga tänkandet utifrån kön? Studenten kan fördjupa sig inom den tidiga feministiska teorin fram till de moderna feministiska vetenskapsstudierna. 


\noindent \textbf{Litteraturförslag}

 \fullcite{}
 
 \fullcite{}
  
 \fullcite{}

\subsection{Kritiska perspektiv}
Detta område behandlar kritiska perspektiv med specifik hänvisning till hur vetenskaplig forskning påverkas av samhälleliga maktfaktorer. Exempel på frågeställningar: Hur förhåller sig vetenskaplig objektivitet till samhällsutvecklingen? Hur är vetenskaplig rationalitet betingad av historiska faktorer? Hur påverkar forskningen samhällets styrning och teknikutveckling? Studenten kan fördjupa sig från det kritiska perspektivets grundverk fram till nutida kritiska perspektiv på vetenskaplig forskning. 


\noindent \textbf{Litteraturförslag}

 \fullcite{}
 
 \fullcite{}
  
 \fullcite{}

\subsection{Vetenskapshistoria (History of science)}
Detta område behandlar historiska perspektiv på vetenskap. Exempel på frågeställningar: Är vetenskaperna kumulativa i sin kunskapsproduktion? Vilken roll spelar socio-historiska faktorer i hur forskningen utformas? Bygger forskning på tidigare forskning? Hur och under vilka omständigheter kan vi betrakta samtida forskning som historisk? I vilken utsträckning påverkas vår samtida syn på forskning och vetenskap av historien? 


\noindent \textbf{Litteraturförslag}

 \fullcite{bachelardNewScientificSpirit1984}
 
 \fullcite{fleckUppkomstenOchUtvecklingen1997}
 
 \fullcite{kuhnLastWritingsThomas2022}
 
 \fullcite{shapinScientificRevolution1996}
  


\section{Modern vetenskapsteori}

\subsection{SSK (Sociology of Scientific Knowledge)}
Detta område behandlar den moderna vetenskapssociologin. Exempel på frågeställningar: Vilken roll spelar vetenskapliga kontroverser för forskningen? Vilka konsekvenser får ett symmetriskt förhållningssätt till vetenskapliga kunskapsanspråk? Hur påverkas forskningen av sociala intressen? Studenten kan fördjupa sig inom den moderna vetenskapssociologin och dess relaterade teoribildningar.


\noindent \textbf{Litteraturförslag}

 \fullcite{}
 
 \fullcite{}
  
 \fullcite{}
 
\subsection{ANT (Actor-Network Theory)}
Detta område behandlar aktör-nätverksteorin. Exempel på frågeställningar: Hur skapas referens genom forskningspraktiker? Hur påverkas vetenskaperna av teknologiska förutsättningar (teknovetenskap)? Vilken betydelse har den nätverksbaserade intressemodellen för hur vetenskaplig kunskap tas fram? Studenten kan fördjupa sig inom den klassiska aktör-nätverksteorin samt senare utvecklingslinjer som  sprungit ur denna.  


\noindent \textbf{Litteraturförslag}

 \fullcite{}
 
 \fullcite{}
  
 \fullcite{}
 
\subsection{Feministiska vetenskapsstudier (Feminist science studies)}
Detta område behandlar teoribildningen feministinska vetenskapsstudier. Exempel på frågeställningar: Hur påverkas forskningen av föreställningar om kön? Vilken roll spelar kön i kunskapsproduktionen? Hur påverkas vetenskaperna av förkroppsligad kunskap? Hur påverkar forskningen samhällets föreställningar om kön? Vilka ojämlikheter reproduceras av vetenskap och teknik? Studenten kan fördjupa sig inom feministiska vetenskapsstudier och ur dessa sprungna vidareutvecklingar. 


\noindent \textbf{Litteraturförslag}

 \fullcite{}
 
 \fullcite{}
  
 \fullcite{}

\subsection{Expertiser (Expertise)} 
Detta område behandlar teoribildningar kring vetenskaplig expertis i demokratiska samhällen. Exempel på frågeställningar: Vad är relationen mellan expertstyre (epistokrati) och folkstyre (demokrati)? Hur konstrueras expertis och de gränser som dras mellan expertis och lekmannaskap? Hur reproducerar expertdrivna processer epistemisk orättvisa? Hur påverkas expertis av samhällsintressen? Studenten kan fördjupa sig inom den litteratur som behandlar vetenskaplig expertis i förhållande till samhället. 


\noindent \textbf{Litteraturförslag}

 \fullcite{}
 
 \fullcite{}
  
 \fullcite{}

\subsection{Forskningspolitik (Science policy)}
Detta område behandlar teoribildningar kring forskningspolitik. Exempel på frågeställningar: Vad kännetecknar forskningspolitiken i förhållande till andra politiska sakfrågor? Vilka historiska utvecklingar av forskningspolitiska inriktningar finns? Hur påverkas forskningspolitiken av andra sociala faktorer? Vilka politiska ideal och målsättningar kommer till uttryck i forskningspolitiken? Studenten kan även fördjupa sig inom angränsande litteraturer som behandlar forskningspolitikens ekonomiska, idéhistoriska och/eller geopolitiska inriktningar.


\noindent \textbf{Litteraturförslag}

 \fullcite{}
 
 \fullcite{}
  
 \fullcite{} 

\subsection{Sociala världar (Social worlds)}
Detta område behandlar teoribildningar kring sociala världar (social worlds theory). Exempel på frågeställningar: Hur skapas kunskap när vetenskaperna möter andra sociala världar? Hur kan multipla sociala världar arbeta tillsammans? Vilken roll spelar enhetliga vs. multipla föreställningar i uppbyggnaden av sociala världar? Vilka relationer mellan forskning och samhälle blir synliga genom sociala världarperspektivet? Studenten kan fördjupa sig både inom nutida teoribildningar och de kanoniserade klassiker som fältet använder som teoretiska resurser. 


\noindent \textbf{Litteraturförslag}

 \fullcite{}
 
 \fullcite{}
  
 \fullcite{}

\subsection{Samproduktion (Co-production)} 
Detta område behandlar teoribildningar kring samproduktion (co-production). Hur kan vetenskap produceras tillsammans med aktörer utanför vetenskaperna? Under vilka villkor skapas kunskap i transdisciplinära sammanhang? Hur kan man förstå relationen mellan vetenskap, politik, teknikutveckling och samhälleliga värderingar? 


\noindent \textbf{Litteraturförslag}

 \fullcite{}
 
 \fullcite{}
  
 \fullcite{}

\subsection{Teknikstudier (Technology studies)}
 Detta område behandlar teoribildningar kring teknikstudier (Technology studies). Exempel på frågor är: Hur påverkar samhället teknikutvecklingen och vice versa? Hur påverkas forskningen av teknikutvecklingen? Hur reproducerar teknologiska innovationer samhälleliga normer? Hur reflekteras samhällelig maktutövning i teknologiska strukturer? Hur påverkar teknologier subjektets självuppfattningar? Studenten kan fördjupa sig inom samtida teknikstudier och dess inriktningar. 
 
 
\noindent \textbf{Litteraturförslag}

 \fullcite{}
 
 \fullcite{}
  
 \fullcite{}












\end{document}